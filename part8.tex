\chapter{メールサーバの冗長化}

メールシステムの冗長化を考えるときは、他のホストからの接続を待つ機能の冗長化が考えられます。
これは、系の全体でアクセスができない時間が無いようにデザインする必要があります。

それと同時に、受信したメールを保存する系を冗長化することができれば、データ保全という意味で、ベストということになります。この二点について、理論と現実について見て行くことにします。


\section{受信メールサーバの冗長化}

メールサーバは、DNSと連携して冗長化します。では、DNSによるホストの冗長化とはどのような概念なのでしょうか。
ここでは、、DNSによるメールサーバの冗長化について考えます。



\subsection{MX優先度による冗長化}

SMTPには、複数のメールサーバによる冗長化が規格として含まれています。これは、MXに優先度を付け、その優先度の順番にメール転送を試みる、もし転送できるメールサーバがなくなれば、メールはincoming queueに一時貯留する、というものです。

この方法の利点は、ロードバランサのような、トランスポート層以下に対応した機器用意しなくても、メールサーバの多重化が可能であることです。また、SMTPの規格にある方法なので、メールサーバの全てが対応可能である、という利点もあります。

優先度が低いメールサーバは、メインのメールサーバの機能が停止している場合に、キューにメールを貯留し、メインのサーバが復活した時メールをフラッシュする、というように、優先度別に役割を変えてサーバを設置できることです。

もし、優先度が同じMXが複数あッたらどうなるでしょうか。この場合は、同じ優先度のMXの中から、統計的な確率でひとつが選択されます。これは、次に説明するDNSラウンドロビンを利用した冗長化と同じです。

\subsection{DNSラウンドロビン}

DNSであるホスト名に対して、複数のAAAAもしくはAで、複数のIPアドレスが記述されている場合、権威DNSは、その問い合わせに対して確率的にどれかひとつのIPアドレスを返します。
このように、DNSエントリの複数のレコードから、ランダムに選択して応答することを、DNSラウンドロビンと言います。

ロビンとは駒鳥のことです。駒鳥が、巣から離れた場所をしばらく巡って追っ手を撒いてから素に戻る様から、このような用語が生まれました。

DNSラウンドロビンでは、、問い合わせはAAAAとAで独立して行われるため、AAAAとAがひとつずつ有るときに、どちらかがランダムに返されるというわけではありません。

DNSラウンドロビンで返されるIPアドレスはランダムに選択されます。そのため、DNSラウンドロビンで選択されうるPostfixのホストは、少なくとも、同等の機能を有する必要があります。この点が、MXの優先度によるメールサーバ冗長化との一番の違いです。

\subsection{DNSによるメールサーバ冗長化の問題}

DNSで確率的なメールサーバの冗長化を計る場合、問題が一つあります。それは、権威DNSからの回答は、k百種リゾルバで一定時間キャッシュされるということです。そのため、同じキャッシュリゾルバを使用するホストに対しては、一定時間同じメールサーバが使用されることとなります。

そのため、メール送信量が多いメールサーバが参照しているキャッシュリゾルバにIPアドレスをキャッシュされた場合、複数あるメールサーバのうち、特定のものだけの負荷が高くなる可能性があります。

このような事象を本質的に防止するには、ロードバランサを使用するという方法が考えられます。ですが、SMTPはステートフルであり、ロードバランサを使用することに対する費用対効果が薄い、という本質的な問題があります。また、ロードバランサを入れるなら、MXの優先度による冗長化を先に検討すべきでしょう。


\section{最終配送先の冗長化}

メールの受信は、DNMの機能で冗長化することができます。ですが、最終は移送先のデータ保存については、ローカルストレージを用いることから、冗長化が難しいという問題があります。

メールデータをバックアップ、リストアするのが、費用対効果面で現実的なのですが、その理由について考えてみます。

\subsection{最終配信先の問題}

Postfixサーバがメールの配信先となるためには、local(8)もしくはvirtual(8)がアクセス可能な、ローカルのストレージが存在する必要があります。
ローカルストレージが必要ということは、最終配信先は冗長化が難しい、ということです。

書き込み先をローカルにマウントしたNASとして、複数のPostfixホストで共有し、書き込みは特定のPostfixサーバだけが行う、という方法も考えられます。
ですが、この方法でもローカルのファイルシステムというい資源を冗長化することは、Postfixの設定のみでは困難です。

また、この場合は、保存を担当するPostfixは、アクティブスタンバイとして、負荷分散は行わず、障害発生時に切り替えを行う方がよいと考えられます。
それは、複数ホストからの同一ファイルシステムへのアクセスによる、ファイルの保安村を防ぐためです。、

最終的なデータ保存場所にも冗長化が必要であれば、Dell EMCなど、冗長化ストレージを利用してハードウェアレベルで冗長化をするなどを検討してください。

\subsection{ソフトウェアによるリモートレプリケーションの問題}

Linuxのlsync(8)とrsync(8)を利用して、保存したメールデータの、リアルタイムなリモートレプリケーションをアプリケーションレベルで行う、という戦略もあります。ですが、筆者は推奨をしません。

これは、メールサーバのパフォーマンスを下げる要員となる、ディスクの読み出しが常時大量に発生するためです。また、CPUの負荷もそれなりにあるため、メールの量によっては、リモート側へのリアルタイム同期が間に合わなくなり、同期が破綻しました。

\subsection{バックアップの取得}

最終保存先を冗長化しなければならないが、高信頼な多重化ハードウェアを導入する予算がないという場合、どのような構成がベターとなるでしょうか。

常時バックアップを取っておき、クラッシュ次はリストアをおこなって復旧を図るのが、その場合の現実的な戦略となります。この戦略を採る場合、メールを保存するファイルシステムは独立したものとして、dump(8)でファイルシステム全体をアックアップします。

dump(8)を用いるのは、特に保存形式としてMaildir/を使用するとき、穂zンしたメールの増減はファイルの増減となるためです。そのため、差分バックアップを利用して、バックアップ量と所要時間を減らすことができるためです。

また、1週間から2週間周期の修正ハノイの塔アルゴリズムを用いることで、バックアップ周期中、どのメールも最低2回はバックアップに含まれるようにすることができます。バックアップアルゴリズムについては、これ以上詳細には立ち入りません。