\chapter{Postfixのストレージ}

Postfixが使用するストレージは、どのような用途ガあるのでしょうか。
また、どのような構成をすれば良いのでしょうか。
本章では、それについて見て行くことにします。

\section{ストレージの用途}

まず、Postfixは、どのような目的でストレージを使用しているかを確認しましょう。それは、incoming queueと、ローカル宛のメールの保存、そして、間接的ですが、ログの出力です。

また、構成によっては、Postfixガイ称するのでありませんが、milterなど協調するプロセスがディスクを脂油している場合もあります。

\subsection{incoming queue}

Postfixが他のメールホストから受け取ったメールデータを一時保存すす領域です。全てのPostfixさーばで、ローカルのファイルシステム上に必要となります。

Postfixは、incoming queueを境にして、主にメールデータを受信して、incoming queueに書き込むプロセスと、incoming queueからメールデータを読み出して、ネクストホップに転送するプロセスと、大きく二種類に分けることができます。

前者をビフォアキュープロセス、後者をアフターキュープロセスという分類をすることもあります。

\subsection{ローカルでのメール保存}

メールの宛先がmain.cf(5)のmydestinationディレクティブと一致するとき、incoming queueから取り出されたメールは、あらためてユーザがアクセスするためのストレージに書き込まれます。
この書き込みは、local(8)もしくはvirtual(8)が担当します。

\subsection{各種テーブル参照}

Postfixテーブル形式のデータを、ハッシュテーブルとしてメールサーバのローカルに置いている場合、ファイルの参照が発生します。ただし、ローカルに置く運用とする場合は、ファイルサイズが大きくないことが多いので、ディスクキャッシュを消極的に期待することもできるでしょう。

これはあくまでもローカルに置いている場合であって、Postfixテーブルを外部データベースとすれば、このディスクI/Oをなくすことが可能です。

\subsection{ログ出力}

Postfixは、syslog(8)経由で、syslog形式のメールログを出力します。ログの出力先は、syslog(8)の設定に依存します。

Postfixでは出力されたログを参照して何かをすることはありません。そのため、syslog(8)の設定で、外部のホストを保存先としても、運用上の問題は生じません。

\subsection{協調プロセスのテンポラリ}

milterやそれ経由で使用するウィルスチェック、spamチェックなどを行うとき、テンポラリにメールや添付ファイルを展開し、その上で読み出して検査をします。

また、Postfixと同じサーバで外部データベースのプロセスを動かしている場合、メールサービスとは別に、そちらのサービスのためのディスクI/Oが生じます。

これらは、ストレージの独立した別のホストで実施することで、Postfixサーバ全体でのディスクI/Oを減らすことが可能です。

\section{ストレージ構成の戦略}

Postfixでは、どのようなディスクI/Oが発生するかについて見て行きました。それを踏まえて、そのようなストレージを構成するか、その戦略を考えていくことにしましょう。

\subsection{UNIX系OSのファイルシステムの性質}

UNIX系のOSでは、ディスクへの書き込みは非同期です。メモリのファイルバッファに保存しておき、なるべきディスクI/O野秋時間を使って書き込みが行われます。

一方、ディスクからの読み出しは、同期的に行われます。つまり、ディスクのリードが発生すると、ファイルシステムが専有されます。また、ファイルのリードは排他的に行われるため、別のプロセスがディスクをしているときに読み出しをリクエストしたプロセスがあれば、またされます。この待ち時間は、ロックが発生してプロセスがまたされている状態となります。

\subsection{ディスクの物理構成}

Postfixだけではなく、全てのメールサーバに言えることですが、パフォーマンスを確保するには、ローカルにおけるディスクのリードを減らすことが重要になります。

ディスクをRAID5か6のアレイとすることで、読み出しに必要案時間を減らすことは、効果があります。
また、費用対効果が許せば、incoming queueとテンポラリをSSDとすることで、パフォーマンスの大幅な向上を見込むことができます。
これは、メモリを追加するなどして、書き込みバッファを増やすよりも直接的な効果がある対策です。

最終配信先であるメールを保存する領域も、incoming queueとは独立した物理ストレージにすることで、排他制御によるロックが発生しにくくします。

また、同一ホスト状でPostfixとmilterの「プロセスを動かす場合は、incoming queueとテンポラリを、それぞれ独立したファイルシステムとして、読み出しの排他制御が発生しにくくする、という方法もあります。


\subsection{メモリファイルシステムの利用と問題}

メモリ容量が許せば、incoming queueをメモリ上のファイルシステムにすることで、ディスクI/Oのパフォーマンスを向上させる方法も考えられます。ですが、筆者はこれを推奨しない考えです。

理由として、incoming queuenお中のメールは、転送に失敗した場合、一定期間の保存とリトライが必要になります。このとき、ホストの再起動やファイルシステムのアンマウントによってデータが失われる、メモリファイルシステムは、用途として向かないということになります。

逆に、milterが利用するテンポラリなどは、メモリファイルシステムでディスクI/Oのパフォーマンスを向上させるのは有効となります。これは、万が一内容が失われても、incoming queue上のメールが残っていれば問題が無いためです。

\subsection{不可避なディスクI/O}

Postfixで不可避なのは、incoming queueの書き込みと読み出しです。incoming queueは到着したメールデータを貯留しておく場所です。そして、メールのネクストホップを決定するために、アフターキュープロセスによる読み出しが行われます。

incoming queueのディスクI/Oだけは、Postfixでは回避することはできません。そのため、パフォーマンス対策を行う場合は、それ以外のディスクアクセスを以下に減らすか、という戦略を採っていく必要があります。

\subsection{回避できるディスクi/O}

あるホスト状で動作するPostfixについて、incoming queue以外は、回避可能です。正確に言えば、別のホストにディスクI/Oを肩代わりしてもらうことができます。

milterなど、協調するプロセスは別のホストで稼動させ、ネットワーク透過でサービスを利用することで、ディスクからの読み出しをなくすことができます。

また、ローカルストレージへの保存を行う専用のメールサーバを用意し、そこにメールを転送する、ログ保存サーバを別途用意する、という方法で、ディスクI/Oを減らすことができます。

ログについても同様で、外部のログサーバでログを保存することにすれば、Postfixが動いているホストでのディスクI/Oを減らすことができます。ただし、ログファイルは書かれた後でPostfixのプロセスが参照することはありません。そのため、外部で保存する優先度は、相対的に低くなります。

それと共に、Postfixテーブルの情報を、メールサーバの外部データベースに参照にすることで、テーブルファイルへのアクセスを減らすことができます。