\chapter{メールのセキュリティ}

\section{メールを受け取るかの判断}

\subsection{ブラックリスト}

ブラックリスト方式は、そのメールサーバで受信を拒否するメールアドレス、ドメイン名を記述しておきます。受信したメールは、全てそのブラックリストと比較し、ブラックリストにあった名前であれば、受信を拒否するという方式です。

Postfixでのブラックリスト方式の受信拒否は、smtpd\_sender\_restrictionsに、条件check\_sender\_restrictionsの値として、Postfixtテーブルを指定します。

\begin{verbatim}
smtpd_sender_restrictions = 
    check_sender_access hash:/path/to/table
\end{verbatim}

ハッシュ形式のPostfixテーブルでは、

\subsection{RBL}

\subsection{S25R}

\subsection{グレイリスティング}

\section{受け取ったメールのチェック}

\subsection{点ぷっふぁいるのウィルスチェック}

\subsection{メールの本文解析}